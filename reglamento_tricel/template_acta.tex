\documentclass{caitemplate}

%\fancyfoot[L]{
    % Left header
 %   \rule[2\baselineskip]{0pt}{0pt}
    % Strut to ensure a 1/4 \baselineskip between image and header rule
  %  \includegraphics[height=3\baselineskip,valign=c]{cai.png}
    % Space
   % \parbox[b]{15cm}{
    %    \centering Centro de Alumnos $\|$ Escuela de Ingeniería $\|$ Pontificia Universidad Católica de Chile \\ (56-2) 2354 4759 $\|$ \textbf{\href{http://www.cai.cl}{http://www.cai.cl}}  $\|$ \textbf{\href{mailto:contacto@cai.cl}{contacto@caiuc.cl}} $\|$ \fecha
     %   } \hfill
%}
\pagestyle{fancy}
\fancyhf{}
\lfoot{%
  \begin{minipage}{\textwidth}
  \parbox{0.17\linewidth}{\includegraphics[height=3\baselineskip,valign=c]{img/cai.png}}\hfill
  \parbox{0.79\linewidth}{
  \centering Centro de Alumnos $\|$ Escuela de Ingeniería $\|$ Pontificia Universidad Católica de Chile\\ (56-2) 2354 4759 $\|$ \textbf{\href{http://www.cai.cl}{www.cai.cl}}  $\|$ \textbf{\href{mailto:contacto@caiuc.cl}{contacto@caiuc.cl}} $\|$ \fecha}\hfill
  \parbox{0.02\linewidth}{\raggedleft \thepage}%
  \end{minipage}
  }

\setlength\parindent{0pt}

\begin{document}

\fancytitle{Instructivo de Campaña Digital}

\textbf{Art. 1°}: Se entenderá por campaña todo acto o manifestación pública que aluda o dé a conocer directamente a un candidato/a, a una lista o a su programa, o bien induzca al voto a una lista o candidato/a, realizada por medios electrónicos oficiales administrados por una lista o candidato/a en competencia, ya sea en un formato de audio, audiovisual, gráfico o de texto.\\

Será responsabilidad del TRICEL en ejercicio definir y dar a conocer cuáles son estos medios electrónicos previo al inicio de la campaña. \footnote{En el marco de establecer los medios electronicos oficiales, el TRICEL no tendrá la facultad de regular cuentas personales de ningún candidato/a en particular.}
\newline
\newline
\textbf{Art 2°}: La campaña será realizada por redes sociales, como Facebook, Instagram, TikTok, WhatsApp o Telegram al tratarse de grupos de difusión administradas por el Centro de Alumnos o cualquier otra plataforma que cada candidato/a o pacto electoral estime conveniente para sí, en cuanto no contravengan las normas pertinentes de este instructivo, la moral, las buenas costumbres y la buena fe.
\newline
\newline
\textbf{Art. 3°}: Todo acto de campaña que implique una comunicación directa o indirecta entre un candidato/a o lista y un elector/a deberá realizarse en día y hora hábil electoral.\\

Los horarios de campaña deberán ser estipulados por el TRICEL en ejercicio y ser informados con antelación a cada candidato/a o pacto electoral.
\newline
\newline
\textbf{Art. 4°}: Todo acto de campaña que implique la recolección de datos personales del público asistente deberá cumplir las leyes vigentes aplicables al uso y protección de datos personales, y no podrá ser comerciada bajo ningún supuesto.
\newline
\newline
\textbf{Art. 5°}: Atendidas las dificultades de comunicación inherentes al contexto de clases en línea, el Centro de Alumnos de Ingeniería difundirá por redes sociales, correo electrónico y su página web todos aquellos antecedentes necesarios para dar a conocer el proceso eleccionario, manteniendo siempre la imparcialidad. Se consideran antecedentes necesarios las fechas del proceso electoral, la explicación del mecanismo de votación y la identidad de los candidatos/as.\\

Asimismo, en su página web, colocará un enlace o referencia a las cuentas de los candidatos/as o la lista en redes sociales y/o a su programa. Será obligación de los candidatos/as suministrar la información necesaria a requerimiento del Centro de Alumnos para cumplir con lo establecido en este artículo respecto a sus redes sociales. El hecho de no aparecer en la página web por no cumplir la obligación anterior no dará lugar a reclamación alguna, pero podrá solicitar tardíamente su inclusión aportando los antecedentes necesarios.
\newline
\newpage
\textbf{Art. 6°}: Queda estrictamente prohibido, ya sea perpetuar por sí o por interpósita persona de parte de las candidaturas en competencia:
\begin{enumerate}
    \item Hacer campaña una vez concluido el período de campaña o en horas inhábiles.
    \item Realizar campaña en contra de algún candidato/a, entendida esta como la difamación, la increpación de mala fe o la afirmación con conocimiento de causa de datos falsos, inexactos o incompletos que induzcan a no votar por él/ella.
    \item Incurrir en o incentivar la ocurrencia de algún tipo de fraude electoral, tales como cohecho, soborno o ciberataques al software de votación.
    \item Todo aquello prohibido por la moral, las buenas costumbres o los Estatutos, la normativa y las resoluciones emanadas del TRICEL, que se adecúe a una campaña digital.
\end{enumerate}
\newline
\newline
\textbf{Art. 7°}: La consumación de las acciones anteriores por los sujetos previamente mencionados serán sancionadas conforme a la normativa aplicable y las penas que se establecen a continuación:
\begin{itemize}
    \item Amonestaciones privadas o públicas.
    \item Obligación de pedir disculpas públicas por medio de la red social que el TRICEL estime conveniente.
    \item Posicionamiento del nombre del candidato/a en último lugar del voto.
    \item Disminución del tiempo en el debate.
    \item Exclusión del candidato/a o la lista del debate.
    \item Suspensión temporal o permanente de redes sociales oficiales de una lista o candidato/a.
    \item Todas aquellas sanciones que considere necesarias atendiendo al caso particular y concreto.
\end{itemize}
\\
El Tribunal aplicará estas sanciones en proporción a la infracción cometida.
\newline
\newline
\textbf{Art. 8°}: El gasto de campaña de cada pacto electoral en caso de las elecciones de Centro de Alumnos y Consejerías Académicas no podrá superar el monto de \$300.000 CLP.
\newline
\newline
\textbf{Art. 9°}: El TRICEL en ejercicio tendrá la facultad de construir nuevos acuerdos con las listas o candidatos/as postulantes, siempre y cuando estos acuerdos estén en concordancia con lo estipulado en este instructivo.
\end{document}